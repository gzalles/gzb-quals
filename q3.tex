\chapter{Immersive Environments: XR in Composition}
\label{ch:xr-mus}

%\textbf{Question three (Shahrokh Yadegari):} once a spatial work is created and recorded, what open XR technologies exist that one can leverage to present their works?

%What are the technical difficulties associated with this kind of work? What are the artistic difficulties associated with this kind of work? 

%What are the differences between WebXR, MobileXR and HMDs (or related technologies like CAVEs)? Be able to talk about the visual system in relation to XR.

%critical lens  

This chapter will be developed with Shahrokh Yadegari. In this chapter we will discuss how open-source developments in XR are allowing more composers to experiment with spatial music and how these multi-sensory systems might be leveraged in the future for the dissemination of their music. We will also discuss some of the new compositional techniques that systems of this sort allow: non-linear sequences, interactivity in general, gamification of musical material, etc. 

We will address the different open frameworks available for the development of these experiences and the technical, and artistic, difficulties associated with these works. We will describe in detail the most popular systems available today for the development of XR experiences and discuss the differences between them in terms of their affordances and limitations.

Namely, we will discuss outline the key trade-offs between presenting spatial music using: HMDs, CAVEs, in concert halls, in gallery spaces, or any other modalities that we might encounter. We are interested in open systems, and especially those that are accessible to those with limited resources (ie. people outside university settings). Finally, we will address some of the fundamental philosophical implications of these systems as they relate to the dichotomy between performer, composer, and audience. 
\section{The History of VR}

\section{contemporary XR techniques}

\subsection{hardware}
\subsubsection{HMDs}
Head mounted displays.

\subsubsection{MOCAP}
MOtion CAPture systems. 

\subsubsection{360\textdegree cameras}

\subsubsection{CAVEs}

%\subsubsection{Panoramic displays}

\section{software}
\subsection{Game engines}
\subsection{WebXR}
\subsection{Mobile XR}
\section{open XR tools }
\section{the future of XR}
\section{conclusion}



