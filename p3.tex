\chapter{Proposal SDY} \label{ch:prop-sdy}
%This is my proposal for the work I would like to pursue over the next few years with Shahrokh's support. It relates to chapter three of this paper. 

\section{Introduction}
%topic/purpose, thesis statement

Spatial music has been a subject of great interest in the electro-acoustic domain for many decades. Ever since Schaeffer, Stockhausen, and Cage began experimenting with tape players as a means of separating the musician from the sound source, composers have ardently explored the possibilities that new technological developments afford them. The last few years have seen an explosion in the field of Extended Reality (XR) with major technology companies rolling out increasingly affordable Head-Mounted Displays (HMD) and game engines like Unity becoming easier to prototype with. 

Unfortunately, there are countless compositions being written today, that are seldom experienced with any spatial information. In our department of music, countless musical works which elegantly make use of octaphonic arrays commonly become down-mixed into stereo, destroying all the spatial information. While the ultimate stereo mix results in a high-quality standardized playback format, the option to document these works with additional spatial information warrants exploring. 

In this proposal, we wanted to explore some of the ways the existing infrastructure of our department, in addition to the vast number of options for presenting spatial audio online, could be leveraged in order to allow for modern representations of our works to be distributed, primarily in an asynchronous fashion. Traditionally, only a small number of individuals, who attend the live event, are privileged enough to listen to these musical works as intended. Our hope is to begin creating a culture of documenting and sharing our spatial works online, using free and open source software and hardware for the dissemination of these works. 

The motivation behind using Free and Open Source Software and Hardware (FOSSH) is to prevent changes in operating system to interrupt the dissemination of these works and also to lessen the financial burden on students and the University. In addition to the economic implications, we appreciate the benefit of sharing these methods with our peers in other institutions whom might be interesting in adopting any of our models to distribute spatial works. The reproducibility of computer music works relies on consistent operating systems and associated hardware, this is part of the motivation for exploring FOSSH. Finally, we believe there is a political message behind this conscious choice - it is a way of diverging from the stronghold of tech giants, relying instead on the global community of computer and electrical engineers which make all this work possible. 

\subsection{From 2D to 3D}

A common problem we encounter in multi-media arts which concern works featuring diffusion is the mismatch between video and audio formats. A common experience for an artist is that of creating a work for a with a 2D video and 3D sound, and finding it difficult, afterwards, to represent these works with 3D audio. Alternatively, we might have spatial audio pieces which do not feature video or any other media. In this case, if we wish to distribute the work using HMD technologies we must decide what, if any, visual representation will accompany our work. 

A complete lack of sensory information might make it difficult to know if/when the piece has begun. A static picture/CGI \footnote{Computer generated imagery: virtual scenes created using frameworks such as \href{https://www.opengl.org/}{OpenGL}.} might be seen as uninspired or lacking in dynamism. All these choices will have artistic implications. 

Generally speaking, there are two common modalities employed by multi-media artists in spatial audio: the movie theatre format, and the HMD format. Each of these can be presented either synchronously or asynchronously, and each have particular deficits and benefits depending on the context. 

The easiest artistic multi-media representation in a synchronous context - that is, when an audience can be present in a multi-channel venue - is the movie theatre format. In this context, a concert venue with a multi-channel sound system and a movie projector is used simultaneously to present an artistic work such as a film or musical experience. The limiting factor here is the number of people which can be present during the performance, while having a proper spatial image, and any geographical barriers. 

In previous concerts organized by the author this movie theatre format was employed to create a few different artistic works. A possible method to distribute these works in a spatial audio supported environment would be using a WebVR framework such as A-Frame to show the 2D video and add additional sources to present spatial audio elements. The micro-movements of the user during viewing should in theory contribute to the sense of immersion.  

\subsection{Interactivity}



\section{Literature Review}
%write in detail about existing solutions to the problem

\section{Methods}
%how would you like to contribute? what could be better?


\section{Results}
%you wont have any results just yet.

\section{Conclusion}
%here is why this is important